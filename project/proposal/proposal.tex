\documentclass[12pt]{article}

\usepackage{amsmath}
\usepackage{amssymb}
\usepackage{amsthm}
\usepackage{enumitem}

\usepackage[legalpaper, margin=1in]{geometry}

\title{ECE 551 \\ Final Project Proposal \\ Nonlinear System Identification with Linear Control}
\author{Alvin Sun}
\date{\today}

\begin{document}

\maketitle

\begin{abstract}
  I propose to build on top of \cite{lusch2018deep}, using an autoencoder-like neural network
  to transform arbitrary nonlinear system into an latent space where the dynamics
  become approximately linear. In addition to the original paper, I propose to extend their
  system identification framework to incorporate external inputs data. Using a similar
  network archiecture, we can learn a set of coordinate system where the control
  laws are also linear, and therefore we can apply well developed linear control algorithms
  with closed form solutions.
\end{abstract}

\section{Motivation}

Control is one of the areas that relate most to our everday lives. It is often time of our
interests to manipulate machines like robotic arms, cars, and even aircrafts. In order
to robustly control those systems, most of the modern control algorithms require the knowledge
of the dynamical system. However, due to practical reasons, many real world dynamical
systems have unknown or even time varying dynamics. Therefore, system identification is
one of the most important areas that enables accurate and robust control. Most of the dynamical
systems are nonlinear in nature, which introduces many difficulties in both identifying
and controlling them. There are quite a few recent work such as \cite{brunton2016discovering}
and \cite{Champion22445} which addresses the identification of nonlinear systems. However,
controlling nonlinear systems in real time remains a hard problem given its
complexity when formulated as numerical optimization problem. As a result, it motivated
me to come up with some method that not only identify nonlinear system dynamis, but also
draw connections to linear control theories.

\section{Project Detail}

In this project, I specifically aim at solving discrete dynamical systems, because digital control
cannot operate in continuous time. In other words, the project is about identifying
and controlling systems that can be characterized by Equation~\ref{eqa:discrete}.
\begin{gather}\label{eqa:discrete}
  x_{k+1} = f(x_k, u_k)
\end{gather}
By Koopman Theory, there exists infinite dimensional linear operators, $\mathcal{A}$ and
$\mathcal{B}$, and some infinite dimensional nonlinear functions $g$ and $h$, such that
\begin{gather}
  g(x_{k+1}) = \mathcal{A} g(x_k) + \mathcal{B} h(x_k, u_k)
\end{gather}
I propose to inherit the autoencoder structure from \cite{lusch2018deep}, but in addition
to approximating for $g$ and $\mathcal{A}$, we also need to estimate for $h$ and $\mathcal{B}$
due to introducing the control inputs. Let $A$ and $B$ denotes the finite dimensional
approximation of $\mathcal{A}$ and $\mathcal{B}$, then $A$ and $B$ can be explicitly learned
while $g$ and $h$ can be approximated using neural networks. Let $z_k = g(x_k)$ and
$v_k = h(x_k, u_k)$ be the states and control inputs in the latent coordinates system. Then
we have an approximated linear dynamical system
\begin{gather}
  z_{k+1} \approx A z_k + B v_k
\end{gather}
Then we can derive control laws $v_k$ by regulating $z_k$ to a desirable terminal states.
Since this latent space dynamics is approximately linear, we can apply linear control
theories to this system. More specifically, I propose to implement a linear
model predictive controller on the latent space, where the optimal control inputs
can be calculated using the Riccati equation.

\section{Project Deliverable}
The project aims at delivering a successful simulation on swing up stabilization of a cartpole
pendulum system with unknown dynamics. If time permits, I would also like to push towards
a cartpole double pendulum swing up controller with unknown dynamics. The latter one is
especially harder due to higher degree of underactuation.

\section{Proposed Timeline}

\begin{description}[align=left, leftmargin=*]
  \item [11/09 - 11/15] \hfill
    \begin{itemize}
      \item Implement infrastructure codes for simulating a variety of dynamics
            models including simple pendulums, cart, cartpole pendulum
      \item Implement infrastructure codes for generating ground truth data with white-noise inputs
      \item Implement autoencoder neural network and setup a data flow / training pipeline
        for system identification process
    \end{itemize}

  \item [11/16 - 11/22] \hfill
    \begin{itemize}
      \item Experiment with different hyperparameters
      \item Compute quantitative metrics for system identification performance
      \item Implement a model predictive controller for simple pendulum swing up
    \end{itemize}

  \item [11/23 - 11/29] \hfill
    \begin{itemize}
      \item Implement linear model predictive control using Riccati equation
      \item Replicate swing up stablization controller using linear control on
        an identified pendulumn
    \end{itemize}

  \item [11/23 - 11/29] \hfill
    \begin{itemize}
      \item Finish up the more complicated cartpole pendulum control.
      \item Final write up and bonus double pendulum control.
    \end{itemize}
\end{description}

\newpage

\bibliography{citations}
\bibliographystyle{ieeetr}

\end{document}
