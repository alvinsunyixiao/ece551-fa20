\documentclass[12pt]{article}
  
\usepackage[margin=1in]{geometry} 
\usepackage{amsmath,amsthm,amssymb,bm,mathtools,scrextend}
\usepackage{fancyhdr}
\pagestyle{fancy}
\usepackage{hyperref}

\usepackage{graphicx}
\usepackage[most]{tcolorbox}
\usepackage{algpseudocode}
\usepackage{algorithm}

\newcommand{\expect}[1]{\mathbb{E}\left[#1\right]}
\newcommand{\av}{\mathbf{a}}
\newcommand{\rv}{\mathbf{r}}
\newcommand{\cv}{\mathbf{cv}}
\newcommand{\C}{\mathbf{C}}
\newcommand{\n}{\mathbf{n}}
\newcommand{\q}{\mathbf{q}}
\newcommand{\x}{\mathbf{x}}
\newcommand{\y}{\mathbf{y}}
\newcommand{\z}{\mathbf{z}}
\newcommand{\0}{\mathbf{0}}
\newcommand{\1}{\mathbf{1}}
\newcommand{\I}{\mathbf{I}}
\newcommand{\X}{\mathbf{X}}
\newcommand{\Y}{\mathbf{Y}}

\newcommand{\hdash}{\rule[.5ex]{1.5em}{0.4pt}}

\newcommand{\DTFT}{\xleftrightarrow{\text{DTFT}}}
\newcommand{\DFT}{\xleftrightarrow{\text{DFT}}}
\newcommand{\ZT}{\xleftrightarrow{\text{ZT}}}


\newcommand{\solspace}{\vspace{3mm} \textbf{Your Solution Here!} \vspace{3mm}}

\begin{document}

\lhead{ECE 551}
\chead{PSET 3 - Linear Systems and Estimation}
\rhead{\today}
 
\section{Some DFT properties}
Fall 2018\\
Let $\x,\y \in \mathbb{C}^N$ be signals with corresponding DFTs $\X, \Y \in \mathbb{C}^N$, and let $C_x \in \mathbb{C}^{N \times N}$ be a circulant matrix, whose first column is $\x$. 
Prove the following
\begin{enumerate}
    \item Time reversal: $x[-n \mod N] \DFT X[-k \mod N]$
    \item Circular convolution theroem (in time and frequency):
        \begin{align*}
            (\x \circledast \y)[n] &\DFT X[k]Y[k] \\
            x[n]y[n] &\DFT \frac{1}{N}(\X \circledast \Y)[k]
        \end{align*}
    \item If $\x$ is a real symmetric signal, namely $x[n] = x[-n \mod N]$, then $\X$ is real.
    \item If $\x$ is a real antisymmetric signal, $x[n] = -x[-n \mod N]$, then $X$ is imaginary.
    \item The eigenvectors of $C_x$ are $w_k[m] = \exp(\frac{2\pi i k}{N}m),$ with eigenvalues $\lambda_k = X[k]$.
\end{enumerate}

\solspace

\pagebreak

\section{Z-Transform of Downsampled Signals}
Fall 2018\\
Let $y[n] = x[Nn]$. Show that the $z$-transform of this downsampled signal satisfies
\begin{equation}
    y[n] = x[Nn] \ZT Y(z) = \frac{1}{N}\sum_{k=0}^{N-1} X(W^k z^{1/N})
\end{equation}
here $X(z)$ is the $z$-transform of $\x$, and $W$ is a primitive $N^{th}$ root of unity, namely $W^N = 1$ and $W^k \neq 1$ for all $0 < k < N$.

Use that result to argue that the DTFT transform downsampling relation is
\begin{equation}
    y[n] = x[Nn]  \DTFT Y(\omega) = \frac{1}{N} \sum_{k=0}^{N-1} X\left(\frac{\omega - 2\pi k}{N} \right)
\end{equation}

\solspace

\pagebreak

\section{Deterministic Correlation and Delay Detection}
Fall 2018\\
Denote the unit delay operator by $\sigma : \mathbb{C}^\mathbb{Z} \rightarrow \mathbb{C}^\mathbb{Z}$ by $()\sigma \x)[n] := x[n-1]$. For $\x, \y \in \ell_2^{\mathbb{Z}}$, define the associated deterministic autocorrelation and crosscorrelation sequences as
\begin{align}
    a_x[k] &:= \langle\x, \sigma^k \x \rangle = \sum_{n \in \mathbb{Z}} x[n] x[n-k]^* \\
    c_{x,y}[k] &:= \langle\x, \sigma^k \y \rangle = \sum_{n \in \mathbb{Z}} x[n] y[n-k]^*
\end{align}

\textbf{(a)} Prove or counter the following statements (note the underlying field is $\mathbb{C}$)
\begin{enumerate}
    \item $a_x[k] = a_x[-k]^*$
    \item $|a_x[k]| \leq a_x[0]$ for every $k \in \mathbb{Z}$
    \item $c_{x,y}[k] = c_{y,x}[-k]^*$
    \item $c_{x,y}[k] = c_{x,y}[-k]^*$
    \item $C_{x,y}(\omega) = X(\omega)Y(\omega)^*$, where $\X,\Y,\C_{x,y}$ are the DTFTs of $\x,\y$ and $\cv_{x,y}$.
\end{enumerate}

\solspace

A signal $\x \in \mathbb{R}^\mathbb{Z}$ whose support is bounded to $[0,...,N-1]$ is received by two antennas, each introduces a different gain and delay. The received signals $\x_1, \x_2$ are given by
\begin{equation}
    x_1[n] = \alpha_1 x_1[n - n_1], \qquad x_2[n] = \alpha_2 x[n-n_2].
\end{equation}
The constants $\alpha_1, \alpha_2 \in \mathbb{R}$ are gain coefficients and $n_1, n_2 \in \mathbb{Z}$ are delays, all unknown.

\textbf{(b)} Based on the result from part a.ii, derive an algorithm to determine the time delay $\Delta := n_2 - n_1$ and the gain ratio $\rho = \frac{\alpha_1}{\alpha_2}$ given inputs $\x_1$ and $\x_2$.

\solspace

\textbf{(c)} Explain why the explicit delay values $n_1$ and $n_2$ cannot be determined, but only their difference $\Delta$. Is the same true for the gains $\alpha_1, \alpha_2$ ?

\solspace

\pagebreak
\section{Interchange of Multirate Operations and Filtering}
Fall 2018\\
Consider the system given by the input output relation
\begin{equation}
    \y = D_2 A D_2 A D_2 A \x
\end{equation}
where $A$ is a convolution filter and $D_2$ downsamples by a factor of 2.

\textbf{(a)} Using the multirate identities, find the simplest equivalent system of the form $\y = D_N H \x$, where $D_N$ is downsampling by $N$ and $H$ is a convolutional filter.
Specify the downsampling factor $N$, and write $H$ in the z-transform and Fourier domains.

\solspace

\textbf{(b)} If $A$ is an ideal half-band lowpass filter, draw the DTFT $H(\omega)$, clearly specifying the cutoff frequencies.

\solspace
\vspace{4cm}

\textbf{(c)} If $A$ is an ideal half-band highpass filter, draw the DTFT $H(\omega)$, clearly specifying the cutoff frequencies.

\solspace

\pagebreak

\section{Computational Problem}
Fall 2018\\
The provided code in \verb|HW3.py| generates two samples $\x_1, \x_2 \in \mathbb{R}^{100}$ of some randomly shifted/scaled waveform based on your UIN (or any other number you input of the same length).

Implement the algorithm you suggested in problem 3 to detect the shift and scale ratio between $\x_1$ and $\x_2$. Plot the input signals, their crosscorrelation, and estimated values, and comment on the results.
Compare your estimated parameters with the true ones.

\solspace


\end{document}